\begin{center}
{\fontfamily{pcr}\selectfont {\LARGE Lời giải bài tập sách "\textbf{TOÁN HỌC CAO CẤP}", }}

\textbf{tác giả {\large {\fontfamily{pcr}\selectfont Nguyễn Đình Trí}}}

\end{center}
\textbf{3.1. Cho}

	$\displaystyle A=\begin{bmatrix}
1 & 3\\
-1 & 2\\
3 & 4
\end{bmatrix} ;A=\begin{bmatrix}
0 & 1\\
3 & 2\\
-2 & 3
\end{bmatrix} ;C=\begin{bmatrix}
2 & -3\\
1 & 2\\
4 & -1
\end{bmatrix}$

Tính 

$\displaystyle  \begin{array}{{>{\displaystyle}l}}
1) \ \ \ \ \ ( A+B) +C;\\
2) \ \ \ \ \ A+( B+C) ;\\
3) \ \ \ \ \ 3A\\
4) \ A^{t} ,B^{t} ,C^{t}
\end{array}$



\textbf{Lời giải:}

1) Theo tính chất của tích phân thì $\displaystyle ( A+B) +C=A+( B+C)$ (giao hoán phép cộng)

Ma trận $\displaystyle A,B,C$ đều là ma trận 3x2 nên ta có $\displaystyle A+B+C=\begin{bmatrix}
1+0+2 & 3+1-3\\
-1+3+1 & 2+2+2\\
3-2+4 & 4+3-1
\end{bmatrix} =\begin{bmatrix}
3 & 1\\
3 & 6\\
5 & 6
\end{bmatrix}$

2) Ma trận $\displaystyle 3A=\begin{bmatrix}
3.1 & 3.3\\
3.( -1) & 3.2\\
3.3 & 3.4
\end{bmatrix} =\begin{bmatrix}
3 & 9\\
-3 & 6\\
9 & 12
\end{bmatrix}$

3) Ma trận chuyển vị của các ma trận $\displaystyle A,B,C$ lần lượt là $\displaystyle \begin{bmatrix}
1 & -1 & 3\\
3 & 2 & 4
\end{bmatrix} ;\begin{bmatrix}
0 & 3 & -2\\
1 & 2 & 3
\end{bmatrix} ;\begin{bmatrix}
2 & 1 & 4\\
-3 & 2 & -1
\end{bmatrix}$



\textbf{3.1. Nhân các ma trận:}

a)$\displaystyle \begin{bmatrix}
3 & 1 & 1\\
2 & 1 & 2\\
1 & 2 & 3
\end{bmatrix} .\begin{bmatrix}
1 & 1 & -1\\
2 & -1 & 1\\
1 & 0 & 1
\end{bmatrix}$

Ta có công thức chung của tích hai ma trận (nếu nhân được) $\displaystyle c_{ij} =\sum _{k=1}^{p} a_{ik} .b_{kj}$

Vì đều là ma trận vuông 3x3 do đó $\displaystyle p=3$. Giả sử ma trận nhân ra có kết quả là $\displaystyle \begin{bmatrix}
c_{11} & c_{12} & c_{13}\\
c_{21} & c_{22} & c_{23}\\
c_{31} & c_{32} & c_{33}
\end{bmatrix}$

Ta tính từng phần tử $\displaystyle  \begin{array}{{>{\displaystyle}l}}
c_{11} =\sum _{k=1}^{3} a_{1k} b_{k1} =a_{11} .b_{11} +a_{12} .b_{21} +a_{13} b_{31} =3+2+1=6\\
c_{12} =\sum _{k=1}^{3} a_{1k} b_{k2} =a_{11} b_{12} +a_{12} b_{22} +a_{13} b_{32} =3.1+1.( -1) +1.0=2\\
c_{13} =\sum _{k=1}^{3} a_{1k} b_{k3} =a_{11} b_{13} +a_{12} b_{23} +a_{13} b_{33} =-3+1+1=-1\\
c_{21} =\sum _{k=1}^{3} a_{2k} b_{k1} =a_{21} b_{11} +a_{22} b_{21} +a_{23} b_{31} =2+2+2=6\\
c_{22} =\sum _{k=1}^{3} a_{2k} b_{k2} =2-1=1\\
c_{23} =1,c_{31} =8,c_{32} =-1,c_{33} =4
\end{array}$

→$\displaystyle AB=\begin{bmatrix}
6 & 2 & -1\\
6 & 1 & 1\\
8 & -1 & 4
\end{bmatrix}$



\textbf{3.3 Thực hiện phép tính sau}

$\displaystyle \begin{bmatrix}
\cos \varphi  & -\sin \varphi \\
\sin \varphi  & \cos \varphi 
\end{bmatrix}^{n}$



Đầu tiên cho $\displaystyle  \begin{array}{{>{\displaystyle}l}}
n=2\rightarrow \begin{bmatrix}
\cos \varphi  & -\sin \varphi \\
\sin \varphi  & \cos \varphi 
\end{bmatrix} .\begin{bmatrix}
\cos \varphi  & -\sin \varphi \\
\sin \varphi  & \cos \varphi 
\end{bmatrix}\\
=\begin{bmatrix}
\cos^{2} \varphi -\sin^{2} \varphi  & -\sin \varphi \cos \varphi -\sin \varphi \cos \varphi \\
\sin \varphi \cos \varphi +\sin \varphi \cos \varphi  & -\sin^{2} \varphi +\cos^{2} \varphi 
\end{bmatrix} =\begin{bmatrix}
\cos 2\varphi  & -\sin 2\varphi \\
\sin 2\varphi  & \cos 2\varphi 
\end{bmatrix}
\end{array}$

Với $\displaystyle n=3$

$\displaystyle  \begin{array}{{>{\displaystyle}l}}
\begin{bmatrix}
\cos 2\varphi  & -\sin 2\varphi \\
\sin 2\varphi  & \cos 2\varphi 
\end{bmatrix}\begin{bmatrix}
\cos \varphi  & -\sin \varphi \\
\sin \varphi  & \cos \varphi 
\end{bmatrix}\\
=\begin{bmatrix}
\cos 2\varphi \cos \varphi -\sin 2\varphi \sin \varphi  & -\cos 2\varphi \sin \varphi -\sin \varphi \cos \varphi \\
\sin 2\varphi \cos \varphi +\cos 2\varphi \sin \varphi  & -\sin 2\varphi \sin \varphi +\cos 2\varphi \cos \varphi 
\end{bmatrix} =\begin{bmatrix}
\cos 3\varphi  & -\sin 3\varphi \\
\sin 3\varphi  & \cos 3\varphi 
\end{bmatrix}
\end{array}$

Giả sử đúng đến $\displaystyle n-1$ tức là $\displaystyle A^{n-1} =\begin{bmatrix}
\cos( n-1) \varphi  & -\sin( n-1) \varphi \\
\sin( n-1) \phi  & \cos( n-1) \varphi 
\end{bmatrix}\rightarrow A^{n-1} .A=\begin{bmatrix}
\cos n\varphi  & -\sin n\varphi \\
\sin n\varphi  & \cos n\varphi 
\end{bmatrix}$

Vậy đúng với $\displaystyle n$ số. Vậy theo giả thiết quy nạp rút được công thức như trên.

\textbf{3.4 Hãy tính }$\displaystyle f( A)$ với $\displaystyle f( x) =x^{2} -5x+3I$ và $\displaystyle A=\begin{bmatrix}
2 & -1\\
-3 & 3
\end{bmatrix}$

$\displaystyle  \begin{array}{{>{\displaystyle}l}}
A^{2} =A.A=\begin{bmatrix}
2 & -1\\
-3 & 3
\end{bmatrix} .\begin{bmatrix}
2 & -1\\
-3 & 3
\end{bmatrix} =\begin{bmatrix}
7 & -5\\
-15 & 12
\end{bmatrix}\\
\rightarrow f( A) =A^{2} -5.A+3I=\begin{bmatrix}
7 & -5\\
-15 & 12
\end{bmatrix} -5.\begin{bmatrix}
2 & -1\\
-3 & 3
\end{bmatrix} +3.\begin{bmatrix}
1 & 0\\
0 & 1
\end{bmatrix}\\
=\begin{bmatrix}
7-10+3 & -5+5+0\\
-15+15+0 & 12-15+3
\end{bmatrix} =\begin{bmatrix}
0 & 0\\
0 & 0
\end{bmatrix}
\end{array}$



\textbf{3.5 Tìm các ma trận cấp 2 mà bình phương bằng ma trận 0}

Gọi ma trận cấp 2 là $\displaystyle A=\begin{bmatrix}
a & b\\
c & d
\end{bmatrix}\rightarrow A^{2} =\begin{bmatrix}
a & b\\
c & d
\end{bmatrix}\begin{bmatrix}
a & b\\
c & d
\end{bmatrix} =\begin{bmatrix}
a^{2} +bc & ac+bd\\
ac+cd & bc+d^{2}
\end{bmatrix} =\begin{bmatrix}
0 & 0\\
0 & 0
\end{bmatrix}$

Suy ra $\displaystyle \begin{cases}
a^{2} +bc & =0\\
ac+bd & =0\\
c( a+d) & =0\\
bc+d^{2} & =0
\end{cases}$

Nếu $\displaystyle c=0\rightarrow a=b=c=d=0$

Nếu $\displaystyle a=-d\rightarrow \begin{cases}
a^{2} +bc & =0\\
ac+bd & =0\\
c( a+d) & =0\\
bc+d^{2} & =0
\end{cases}\rightarrow \begin{cases}
a^{2} +bc & =0\\
ac-ba & =0\\
bc+a^{2} & =0
\end{cases}\rightarrow \begin{cases}
a^{2} +bc & =0\\
a( c-b) & =0
\end{cases}$

Nếu $\displaystyle a=0\rightarrow \left[ \begin{array}{l l}
b & =0\\
c & =0
\end{array} \right.$

Nếu $\displaystyle c=b\rightarrow a^{2} +b^{2} =0\rightarrow a=b=0$